\section{Bài toán tối ưu tuổi thọ trong mạng cảm biến không dây ngầm}
\begin{frame}[noframenumbering]
    \frametitle{Nội dung trình bày}
    \tableofcontents[currentsection]
\end{frame}

\begin{frame}
    \frametitle{Định nghĩa tuổi thọ}
    \begin{itemize}
        \item Tuổi thọ của mạng bắt đầu từ khi hoạt động cho đến khi có một nút ngừng hoạt động.
        \item Kí hiệu: $T_n^n = \min_{v \in V} T_v$
        \item[] Trong đó $T_v$ là thời gian sống của nút $v$.
    \end{itemize}
\end{frame}

\begin{frame}
    \frametitle{Mô hình bài toán}
    \textbf{Đầu vào }
    \begin{itemize}
        \item Không gian 3 chiều cần khảo sát với chiều dài và chiều rộng lần lượt là $W$ và $H$, chiều cao (sâu) vô hạn.
        \item $S = \{s_1, s_2,…, s_n\}$: tập các nút cảm biến được triển khai, mỗi nút cảm biến $s_i$ có các thuộc tính:
        \begin{itemize}
            \item[] $(xs_i, ys_i)$: tọa độ trên mặt $Oxy$
            \item[] $hs_i$: độ cao so với mực nước biển
            \item[] $r$: bán kính truyền thông 
            \item[] $l$: số bit mà nút cảm biến gửi tới trạm cơ sở
        \end{itemize}
        \item $F = \{f_1, f_2,…, f_m\}$: tập các vị trí khả thi để triển khai các nút chuyển tiếp, $f_i$ gồm các thuộc tính:
        \begin{itemize}
            \item[] $(xr_i, yr_i)$: tọa độ trên mặt $Oxy$
            \item[] $hr_i$: độ cao so với mực nước biển
            \item[] $R$: bán kính truyền thông 
        \end{itemize}
    \end{itemize}
\end{frame}

\begin{frame}
    \frametitle{Mô hình bài toán}
    \begin{itemize}
        \item $D = \{d_{11}, d_{12},…, d_{nm}\}$: ma trận khoảng cách 
        \begin{itemize}
            \item[] $d_{ij}$: khoảng cách từ sensor $s_i$ đến relay $f_j$
        \end{itemize}
        \item $C = \{c_{11}, c_{12},…, c_{nm}\}$: ma trận kết nối
        \begin{equation} 
            c_{ij} = \begin{cases}
                1 & \textrm{nếu $d_{ij} \leq r_i + R_j$}\\
                0 & \textrm{nếu ngược lại}
            \end{cases}
            \label{eqn:simple_one} 
        \end{equation}     
        \begin{itemize}
            \item[] Ý nghĩa: sensor $s_i$ có thể kết nối với relay $f_j$ nếu khoảng cách giữa hai nút không vượt quá tổng bán kính truyền thông của chúng
        \end{itemize}
        \item Năng lượng tiêu hao khi truyền $l$ bits dữ liệu từ sensor $s_i$ đến relay triển khai tại $f_j$
        \begin{equation}
            Et_{ij} = l * (E_{TX} + e_{fs} * d_{ij}^2) 
            \label{sensor_consumption}
            \footnote{W. R. Heinzelman, A. Chandrakasan, H. Balakrishnan, Energy-efficient communication protocol for
            wireless microsensor networks, in: Proceedings of the 33rd annual Hawaii international conference on
            system sciences, IEEE, 2000.}
        \end{equation} 
    \end{itemize}
    
\end{frame}

\begin{frame}
    \frametitle{Mô hình bài toán}
    \begin{itemize}
        \item Năng lượng tiêu hao của relay triển khai tại $f_j$ nhận dữ liệu từ $x_j$ sensors, tổng hợp và gửi tới trạm cơ sở
    \begin{equation}
        Er_j = l * (x_j * E_{RX} + x_j * E_{DA} + e_{mp} * d_{jtoBS}^4)
        \label{relay_consumption}
        \footnote{W. R. Heinzelman, A. Chandrakasan, H. Balakrishnan, Energy-efficient communication protocol for
            wireless microsensor networks, in: Proceedings of the 33rd annual Hawaii international conference on
            system sciences, IEEE, 2000.}
    \end{equation}
    \item $E_{max}$: năng lượng tiêu thụ tối đa mà một nút trong mạng có thể đạt đến
    \end{itemize}
\end{frame}

\begin{frame}
    \frametitle{Mô hình bài toán}
\textbf{Đầu ra }
\begin{itemize}
    \item $z = (z_1, z_2,…, z_m)$: vector quyết định
    \begin{equation}
        z_j = \begin{cases}
            1 & \textrm{nếu có relay triển khai tại $f_j$}\\
            0 & \textrm{nếu ngược lại}
        \end{cases}
    \end{equation}
    \item $A = \{a_{11}, a_{12},…, a_{nm}\}$: ma trận quyết định
    \begin{equation}
        a_{ij} = \begin{cases}
            1 & \textrm{nếu $s_i$ kết nối tới $f_j$}\\
            0 & \textrm{nếu ngược lại}
        \end{cases}
    \end{equation}
\end{itemize}

\end{frame}

\begin{frame}
    \frametitle{Mô hình bài toán}
\textbf{Ràng buộc }

\begin{equation}
    \sum_{j = 0}^m a_{ij} = 1 ~\forall i = 0, 1, \ldots, n: \textrm{mỗi sensor chỉ kết nối tới 1 relay}
\end{equation}
\begin{equation}
    z_j = a_{1j} \lor a_{2j} \lor \ldots \lor a_{nj} ~\forall j = 1, 2, \ldots, m
\end{equation}

\textbf{Hàm mục tiêu }

\begin{equation}
    \frac{\alpha}{m} * \sum_{j = 1}^m z_j + \frac{1 - \alpha}{E_{max}} * E_x \rightarrow min
    \label{obj_func}
\end{equation}

Trong đó:
\begin{itemize}
    \item $E_x$ là năng lượng tiêu  hao lớn nhất của một nút trong lời giải. 
    \item $0 \leq \alpha \leq 1$: trọng số đánh giá độ quan trọng của số nút chuyển tiếp được sử dụng.
\end{itemize}
    
\end{frame}

\begin{frame}
    \frametitle{Hướng tiếp cận}
    \begin{itemize}
        \item Giải thuật chính xác.
        \item Các giải thuật xấp xỉ.
    \end{itemize}
\end{frame}